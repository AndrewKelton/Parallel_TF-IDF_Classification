\documentclass[conference]{IEEEtran}
\IEEEoverridecommandlockouts
% The preceding line is only needed to identify funding in the first footnote. If that is unneeded, please comment it out.
\usepackage{cite}
\usepackage{amsmath,amssymb,amsfonts}
\usepackage{algorithm}
\usepackage{algpseudocode}
% \usepackage{algorithmic}
\usepackage{graphicx}
\usepackage{textcomp}
\usepackage{xcolor}
\usepackage{hyperref}
\usepackage{caption}
\usepackage{capt-of}
\usepackage{listings}
\usepackage[T1]{fontenc}
% \usepackage{courier} % or \usepackage{inconsolata} for a modern look


% Set monospaced font just for the algorithm block
% \newcommand{\algorithmfont}{\fontfamily{pcr}\selectfont}

% Replace assignment arrow with equal sign
% \algrenewcommand{\algorithmicset}{\algorithmfont :=}  % only affects := (not ←)
% \algrenewcommand{\algorithmicwhile}{\algorithmfont while}
% \algrenewcommand{\algorithmicdo}{\algorithmfont do}
% \algrenewcommand{\algorithmicfor}{\algorithmfont for}
% \algrenewcommand{\algorithmicif}{\algorithmfont if}
% \algrenewcommand{\algorithmicthen}{\algorithmfont then}
% \algrenewcommand{\algorithmicelse}{\algorithmfont else}
% \algrenewcommand{\algorithmicend}{\algorithmfont end}
% \algrenewcommand{\algorithmicfunction}{\algorithmfont function}
% \algrenewcommand{\algorithmicreturn}{\algorithmfont return}

% Redefine \State to manually replace ← with =
% \usepackage{xparse}
% \RenewDocumentCommand{\State}{o+m}{%
%   \Statex
%   \algorithmfont
%   \IfValueTF{#1}{\algorithmicindent\textbf{#1}\ }{}%
%   \StrSubstitute{#2}{←}{=}\unskip%
%   \par
% }

\lstdefinestyle{ieeecpp}{
  basicstyle=\ttfamily\footnotesize,
  keywordstyle=\color{blue},
  commentstyle=\color{gray},
  stringstyle=\color{red},
%   numbers=left,
%   numberstyle=\tiny\color{gray},
%   stepnumber=1,
%   numbersep=5pt,
  backgroundcolor=\color{white},
  frame=single,
  breaklines=true,
  breakatwhitespace=false,
  tabsize=2,
  captionpos=b,
  showstringspaces=false
}
% \sloppy

\newcommand{\code}[1]{\lstinline[basicstyle=\ttfamily]|#1|}
\def\BibTeX{{\rm B\kern-.05em{\sc i\kern-.025em b}\kern-.08em
    T\kern-.1667em\lower.7ex\hbox{E}\kern-.125emX}}

\begin{document}
\title{A Parallel Term Frequency---Inverse Document Frequency Implementation in C++} %\\

% \thanks{Identify applicable funding agency here. If none, delete this.}
% }

\author{\IEEEauthorblockN{Andrew Kelton}
\IEEEauthorblockA{\textit{Dept. of Computer Science} \\
\textit{University of Central Florida}\\
Orlando, USA \\
an597152@ucf.edu}
}

\maketitle

\begin{abstract}
    TF-IDF (Term Frequency-Inverse Document Frequency) is a numerical statistic that represents the significance of a word in a document relative to its corpus (collection of documents) it is included in. It is a widely used algorithm used for prepro-cessing in text classification. However, its normal, sequential, algorithm takes quite a long time to vectorize and compute TF-IDF scores across an entire corpus of documents. There are also not many TF-IDF implementations in C++, the most wide-ly used TF-IDF implementations are in Python,  while popular for prototyping and data analysis, is generally slower than C++ for computational tasks. In this project, I present a paral-lel TF-IDF vectorization, computation, and classification im-plementation written in C++. The goal of this project is to pro-vide an efficient and quick parallelized TF-IDF vectorizer and classifier in C++, and to compare my parallel implementation to my sequential implementation. I was successful in creating the parallel TF-IDF vectorization, computation, and classification. Source code is included in this project, and the performance evaluation in Python is included as well. 
\end{abstract}

\begin{IEEEkeywords}
    C++, classification, computation, efficient, parallel, Python, quick, sequential, Term Frequency-Inverse Document Frequency (TF-IDF), vectorization
\end{IEEEkeywords}

\section{Introduction} \label{sec:introduction}

Natural language processing (NLP) is a vital field in today’s world. One of the key techniques used in NLP is Term Frequency-Inverse Document Frequency (TF-IDF), which helps identify important words within a document, while filtering out less relevant words. Search engines such as Google utilize TF-IDF to rank web pages, filtering billions of somewhat relevant words to ensure users receive the most meaningful results. However, computing TF-IDF at this scale is extremely computationally expensive, especially when processing large datasets, such as internet search results.

A common approach to handling this challenge is to precompute and store TF-IDF scores in a database. However, maintaining such a vast database is resource intensive. An alternative solution is to compute TF-IDF dynamically when needed. Unfortunately, most existing implementations of TF-IDF are sequential and written in Python~\cite{b4}, which limits their efficiency and performance, especially for large-scale applications.

This project focuses on implementing a parallelized TF-IDF classifier in C++ to significantly improve computation speed and efficiency. It has been shown that C++ can be up to 100 times faster than Python for certain computational tasks due to its optimized memory management and lower-level control over hardware resources. By leveraging parallel computing, this implementation aims to further enhance TF-IDF performance by distributing the workload across multiple processing cores.

Beyond search engines, TF-IDF is widely used in spam detection, sentiment analysis, and document clustering, where faster computation can lead to real-time processing improvements. Traditional implementations suffer from bottlenecks in matrix operations, memory usage, and sequential execution, which can be mitigated using optimized C++ techniques. %This project will integrate the Armadillo library, a highly efficient linear algebra library, to handle matrix operations and accelerate TF-IDF computation.

By implementing TF-IDF classification in C++ with parallelization, this project aims to demonstrate significant performance improvements over traditional approaches. The expected outcomes include faster execution times, efficient memory usage, and scalability for large datasets, making this approach highly suitable for real-world NLP applications.

\subsection{Problem Statement}
\textbf{I have achieved the main goal for this project: Implement a parallel Term-Frequency Inverse Document Frequency (TF-IDF) text classifier in C++ and benchmark it against my sequential implementation of a TF-IDF text classifier in C++.}
I will be using the sequential TF-IDF implementation as my control. As far as I know, there is no publicly available parallel TF-IDF implementation in C++. With this project I seek to be able to implement key components of the commonly used text vectorization and classification algorithm into an efficient parallelized, C++ implementation. Once I have equivalent implementations (sequential and parallel), I will run tests on the two different implementations to see if there are any performance, efficiency, accuracy, or precision differences between them. I will be looking mainly at speed and efficiency, and briefly discussing text classification accuracy and precision.

\section{Motivation} \label{sec:motivation}
In my research for this project, I could not find any parallel TF-IDF implementations for articles or text written in C++, either in literature or production. It surprised me that I could not find any strictly parallel TF-IDF implementations for text available in C++. While I would have appreciated the opportunity to test my parallel TF-IDF algorithm against others that are widely used, no such implementation appears to exist. The opportunity to review a similar implementation could have given insight on the key differences between a strictly parallel TF-IDF in C++ and user-level parallel TF-IDF techniques normally found in Python. Although this lack of existing work is disappointing, it confirms the novelty and significance of this project.

An explanation for this gap in C++ implementations, is the dominance of existing Python-based data science and machine learning libraries. Libraries such as Scikit-learn in Python have created specific functions that allow users to implement parallelism in their code, however, they do not provide a strictly parallel TF-IDF implementation~\cite{b4}. These implementations have often been criticized for not being thread-safe and not providing full control for concurrency. This requires users to explicitly optimize and control shared memory, while maintaining thread-safety. In contrast, C++ offers low-level access and control over parallel processing, making it a strong candidate for developing high-performance, large-scale text classification solutions

The absence of scalable parallel  TF-IDF implementations in C++ also suggests an opportunity to contribute a scalable, efficient solution that can be used freely by C++ programmers. This work aims to bridge that gap, demonstrating the feasibility and benefits of parallelizing TF-IDF in C++ for text vectorization and classification.

In the future, I plan to incorporate more functions that allow users to view more data resolved from the TF-IDF vectorization.

\section{Related Work}
As explained above, TF-IDF is a widely used technique to evaluate the importance of words within a document. Although I could not find any parallel Tf-IDF implementations in C++, I was able to find research in relation to using cosine similarity for text classification. I relied on the findings of Park, Hong, and Kim's research on the use of cosine similarity for text classification~\cite{b2}. Their research consisted of utilizing cosine similarity to enhance traditional classifiers, however, I solely used cosine similarity to classify testing documents in this project. I did however use their~\cite{b2} cosine similarity function as a base to my own. I will discuss the specific aspects in which Park, Hong, and Kim's cosine similarity implementation differs from my own in the next section. 

\subsection{An Overview of TF-IDF}
The following is the mathematical equation of the TF-IDF algorithm, where the TF-IDF score for word \textit{t} in the document \textit{d} is defined as~\cite{b7}:
\[
\text{TF-IDF}(t, d) = \text{TF}(t, d) \times \text{IDF}(t)
\]
Where
\[
TF(t,d) = \frac{\text{number of times } t \text{ appears in } d}{\text{total number of terms in } d}
\]
\[
IDF(t) = \log \left( \frac{N}{1 + \text{freq}(t,d)} \right)
\]
This is the same algorithm and mathematical equation used to calculate TF-IDF scores in this project.

\section{Implementation} \label{sec:implementation}
The following is the breakdown of the tasks I completed on this project:

\begin{itemize}
    \item Write the C++ implementation for reading, splitting, and extracting document data from CSV files
    \item Implement modules containing classes and namespaces for the representations of:
    {
        \begin{itemize}    
            \item TFIDF
            \item Corpus
            \item Document
            \item Category
        \end{itemize}
    }
    \item Implement tasks for preprocessing data:
    {
        \begin{itemize}    
            \item Skip stop words (common words such as the, is, etc. that do not contribute importance to a document)
            \item Remove punctuation
            \item Convert all letters to lower case
            \item Stem words using the \code{english_stem} function from the Oleander Stemming Library~\cite{b5}
        \end{itemize}
    }
    \item Implement the functions to vectorize a corpus of documents.
    \item Implement the functionality TF-IDF computation in C++.
    \item Implement the functionality text classification using cosine similarity in C++.
    \item Create a testing environment for extracting benchmarks, performance, accuracy, and precision results
    {
        \begin{itemize}    
            \item Develop Python and Bash scripts to automate extraction comparison between the two implementations and generate human readable results.
        \end{itemize}
    }
    \item Compare benchmarks between the two implementations
    {
        \begin{itemize}    
            \item Measure performance amongst sections
            \item Measure classification accuracy and precision
            \item Test multiple testing corpuses with training corpuses
        \end{itemize}
    }
\end{itemize}

\subsection{Plan for Implementation}
To summarize the above sections, my goal is to implement a parallel TF-IDF vectorizer and classifier for text processing in the C++ programming language. I plan to compare both my C++ parallel and sequential implementations. I am primarily interested in how my parallel implementation compares to the sequential implementation with regards to performance and accuracy, but I am also interested to see if any major changes to design between the two are essential. The goal is to provide a parallel C++ TF-IDF implementation for use in general NLP programming. My goal is to allow users to use my library without using C++’s sequential TF-IDF from mlpack~\cite{b3} or switching to Python. I first began by creating the structs and classes for the Corpus, Document, and Category objects and the descriptors for their tasks and uses. The following are the three main sections of the program that will differ significantly between the sequential and parallel implementation.

\begin{itemize}
    \item Vectorization
    \item TF-IDF Computation
    \item Categorization
\end{itemize}

When using a training corpus, I record performance benchmarks for each section. For example, the timer starts when the Vectorization section begins and ends when the Vectorization section completes. For an test corpus, I record performance benchmarks as a whole, including all sections and classification time in one recording. Execution times are recorded in milliseconds and begin and end recording in the same blocks in both implementations.

After all sections complete, including classification (for an test corpus), I record accuracy and precision of the classification for each document classified. I compare execution times of each section, classification accuracy, and classification precision between the parallel and sequential implementations. 

\subsection{Classification of Text}
The testing data's text is classified using cosine similarity of the TF-IDF scores of a test data's text, $d_i$, and a training data's category's combined TF-IDF scores using Algorithm 1, $c_j$, whose TF-IDF scores have already been calculated~\cite{b1} and $d_i$ is a document in the test corpus and $c_j$ is the entire \code{TFIDF} of normalized TF-IDF scores of all documents in a \code{Category}. The cosine similarity approach used in this work is largely based on the methodology outlined by Park, Hong, and Kim in \cite{b2}, due to the findings of cosine similarity's significance in using TF-IDF scores to classify text. The approach presented in this work differs from that of Park, Hong, and Kim in that it computes the cosine similarity between a single document and a combined TF-IDF score vector for a category, rather than computing cosine similarity across all documents. Park, Hong and Kim's approach is given by:
\begin{equation}
\begin{aligned}    
\text{\textit{Sim}}(doc1, doc2) &= \frac{doc1 \cdot doc2}{\|doc1\| \|doc2\|} \\
                       &= \frac{\sum_{i=1}^{n} A_i B_i}{\sqrt{\sum_{i=1}^{n} A_i^2} \sqrt{\sum_{i=1}^{n} B_i^2}}
\end{aligned}
\end{equation}
where $A_i$ and $B_i$ represent the components of vectors \textit{doc1} and \textit{doc2}, respectively~\cite{b2}.

While the cosine similarity equation used in this report is as follows: 
\begin{equation}
\begin{aligned}
\text{\textit{Sim}}(d_i, c_j) &= \frac{d_i \cdot c_j}{\|d_i\| \|c_j\|} \\
                     &= \frac{ \sum\limits_{t \in d_i \cap c_j} d_i(t) \cdot c_j(t) }
        { \sqrt{ \sum\limits_{t \in d_i} d_i(t)^2 } \cdot \sqrt{ \sum\limits_{t \in c_j} c_j(t)^2 } } \label{eq:cosine-sim} \\
% &\text{where } \|d_i\| > \varepsilon \text{ and } \|c_j\| > \varepsilon \nonumber
\end{aligned}
\end{equation}
where $d_i(t)$ and $c_j(t)$ denote the TF-IDF weights of term $t$ in document $d_i$ and category $c_j$, respectively. The summation $\sum_{t \in d}$ iterates over all terms $t$ in the given document, and the intersection $d_i \cap c_j$ includes only the terms shared by both documents. The threshold $\varepsilon = 10^{-9}$ ensures stability by avoiding division by zero when either document has no weighted terms.

The methodology of classifying an test document is as follows:
\begin{equation}
    \hat{y} = \arg\max_{1 \leq j \leq 5} \textit{Sim}(d_i, c_j)
\end{equation}
where it iterates through all 5 indices of \code{Category} object vector and saves the largest cosine similarity value each time, as that will be related to its classifed category. The equation iterates through 5 indices of the \code{Category} object vector since this project is currently limited to 5 classification categories that are stated in~\cite{b6}. The following classification categories are supported in this project in sequential order:
\begin{enumerate}
    \item Sports
    \item Business
    \item Politics
    \item Tech
    \item Entertainment
\end{enumerate}
    

% To find the most similar category to an test (uncategorized) document, I iterate through all \code{Category} objects in the \code{std::vector} of categories. In this report, I used 5 category types:


\subsection{Anticipated \& Encountered Challenges}
I anticipated the following major challenges with implementing this project: C++’s non-memory safe design, the lack of similar multi-threaded implementations, and my lack of experience with C++.

Since I used this project as an opportunity to learn C++, I anticipated some difficulty would come from learning the language, however, I was quite captivated by C++’s resemblance of Java and C. This is my first major project I have undergone in C++, and I have learned a lot about its object-oriented programming (OOP) capabilities. I became quite acquainted with the language’s strengths, limitations, and code design.

C++’s design prioritizes performance and flexibility, but this comes at the cost of memory safety and built-in concurrency primitives. Though these traits allow fine-grained control over system resources, they also introduce challenges when implementing a multi-threaded system. A major issue I encountered was managing shared data across threads. Unlike higher-level languages with built-in memory safety mechanisms, C++ requires explicit handling of synchronization and resource management. Simply sharing a data structure across multiple threads is not straightforward, as concurrent access must be carefully controlled to prevent data races and undefined behavior.

One common approach is to use \code{std::mutex} to lock shared resources, ensuring thread safety at the cost of potential contention. However, using mutexes alone is not always ideal for high-performance applications, as they introduce blocking behavior. An alternative is \code{std::atomic}, which enables lock-free programming by allowing atomic operations on shared data. Despite these tools, I found that designing an efficient and safe multi-threaded system in C++ requires a deep understanding of low-level memory management and synchronization mechanisms. Thankfully, my extensive experience with the C programming language in systems programming, proved quite helpful when coming across sections that required meticulous memory management.

However, this is just one example of the increased complexity required to implement a parallel TF-IDF while adhering to C++’s memory management requirements. Encountering more situations like this was my greatest concern for this project. In the worst case, I relied on C-style programming techniques. These tools are designed to avoid undesired results such as garbage values or invalid pointers.

\subsection{More Challenges \& Changes to Design}
When testing the code, I noticed it was quite difficult to understand exactly where errors where occurring, even after implementing \code{try: catch:} blocks within the source code. To mitigate these issues and increase ease of usability, I decided to create the \code{TFIDF} namespace and \code{TFIDF}\_ class.

The \code{TFIDF}\_ class is a user-side class that takes input on numerous variables, such as: number of threads to use, print output, print errors, file inputs, and file outputs, just to name a few. This greatly increased the code-readability, reusability, and effectiveness of recording metrics amongst multiple inputs. The class contains three functions:
\begin{itemize}
    \item \code{process_training_data} %\lstinline[language=C]|process_trained_data|
    \item \code{process_testing_data}% \lstinline[language=C]|process_un_trained_data|
    \item \code{process_all_data}% \lstinline[language=C]|process_all_data|
\end{itemize}

These functions allow the user to either process training and test data separately, or process training data then process test data by only calling one function. When testing the code with multiple datasets and varying thread counts, \code{process_all_data} is called to provide a standard for code usage and recording of benchmarks.

Another major issue I became aware of when first testing the program, was the extreme inaccuracy of category classification when utilizing multiple threads. This issue was mainly due to the fact I was not ensuring all threads maintained the same data across all threads. I would lock the \code{vector} that contains the final list of \code{Category} objects when pushing a new \code{Category} but would not lock or ensure proper synchronization when categorizing data. Therefore, \code{Category} objects contained incorrect document indexes, since their chosen index had changed due to another thread modifying the \code{vector} of documents. I was able to fix this issue by locking the entire program when a thread is actively creating a \code{Category} object. However, this solution is not efficient and allows for long-blocking periods, making it essentially sequential with a minor decrease in performance, due to thread creation, deletion, and idle-waiting. In the future, I hope to implement a queue and worker threads to efficiently dequeue \textit{Document} objects and classify them to their correct category without inefficient blocking or waiting.

\subsection{Synchronization}
The approach for my implementation requires support of locking mechanisms and atomic operations. To avoid data races, whenever pushing or emplacing to a \code{TFIDF}, I lock the \code{TFIDF} and ensure only one thread updates the shared \code{TFIDF} at a time. This design guarantees only one thread updates the \code{TFIDF} at a time, while also ensuring a short critical section to avoid long blocking. Atomic operations are used when classifying our test data and incrementing the total and correct count of classifications. This approach guarantees all threads contain the same data throughout the entire TF-IDF process.

My implementation also ensures proper thread balancing, avoiding scenarios where some threads have completed their tasks, while others have just begun. This is achieved by splitting the dataset into roughly equal chunks based on the number of available threads we are testing with as seen in Algorithm 2 . Each thread is assigned a portion of data to: vectorize, compute TF-IDF scores, and classify test data.

Additionally, thread synchronization is cautiously managed to ensure threads are not being blocked by long critical sections. The task distribution is designed so that no thread holds up others for extended periods of time. Immediately after updating a critical section that uses locks, threads resume their tasks and remain in sync containing the same data. This approach not only ensures efficient workload, but also efficient use of resources, minimizing idle time and maintaining overall performance.

\section{Testing} \label{sec:testing}
To accurately compare the results against the sequential C++ implementation, I used the same original dataset~\cite{b6} and split it into 80\% training data and 20\% testing data. This ensured consistent input across all versions of the program. As discussed in Section~\ref{sec:implementation}, I measured execution time in milliseconds and recorded the classification accuracy and precision. Below I will present the results of performance, accuracy, and precision analysis.

\subsection{Testing Environment}
All code was written, compiled, and tested using C++17 on the University of Central Florida’s high-performance Secure Shell Protocol (SSH) server, \textit{eustis3}. Eustis3 runs Ubuntu 22.04 LTS with Linux kernel version 5.15.0 and is equipped with dual-socket Intel\textsuperscript{\textregistered} Xeon\textsuperscript{\textregistered} Silver 4216 CPUs, totaling 64 threads (2 sockets × 16 cores × 2 threads per core). The system provides 187~GiB of RAM, of which approximately 58~GiB were free and 75~GiB available during testing. This setup offered sufficient parallel processing capabilities for benchmarking the proposed algorithm while maintaining a consistent Linux-based development environment to minimize OS-specific inconsistencies.

The main test files were written to be mostly one-to-one from each other. The only differences in the main test files are the inputs when creating the \code{TFIDF}\_ object. The tests are conducted by inputting the respective number of threads for the \code{TFIDF}\_ object to handle. All testing files use the same datasets: \textit{dataset-1}, \textit{dataset-2}, and \textit{dataset-3}, which consist of collections of categorized BBC News articles of varying lengths and complexities. These datasets were sourced from \cite{b6}

The runner is a Makefile which compiles and executes the parallel implementation, then the sequential implementation. The command make test will compile and execute both the main test files with the preset training data file and output the data into its respective files in the test-outputs folder.

To run multiple tests of all numbers of threads and the sequential test, I used a Bash script. The script runs each testing scenario ten times before testing the next scenario. This allows shorter testing scenarios to run and complete before testing longer scenarios, such as the sequential implementation, which may take upwards of twenty minutes to complete per test. Figure 1 is an example of what a block in the Bash script looks like when testing the code with two threads. This script was used to gather ten iterations of each test across the three datasets.

\begin{figure}[htbp]
\centering    
\begin{minipage}{0.95\linewidth}
\begin{lstlisting}[style=ieeecpp, language=bash]
NUM_ITERATIONS=10

for ((i=1; i<=$NUM_ITERATIONS; i++)); do
    make 2-test
done
\end{lstlisting}
\captionof{figure}{\footnotesize Example block of bash testing script to test 2 threads.}
\end{minipage}
\end{figure}

\section{Evaluation} \label{sec:evaluation}
Once both the sequential and parallel TF-IDF implementations were finished in C++, I could analyze the differences in performance, accuracy, and precision between the two. For this paper, I will be using the results from dataset-3, which is the largest dataset used and the most likely to be used in a real-world scenario.

\begin{figure}[htbp]
\centerline{\includegraphics{Fig2.png}}
\caption{Speedup results from 1--1024 threads.}
\label{fig}
\end{figure}

In figure 2, there is a dramatic change in performance from 1 to 64 threads, and then it fizzles out, remaining close to or worse than before. However, the Categories performance remains similar amongst all threads. This is due to the way the Categories are updated between the sequential and parallel implementations. As stated earlier, the parallel implementation for the Categories section is essentially sequential by locking the entire corpus object before completing categorization tasks. In the future, I hope to fix this issue by implementation a conccurent queue for categorization tasks.


Figure 3 illustrates the mean classification accuracy of 10 iterations across all tests. As we can see, there is a dramatic drop in accuracy between 2 and 32 threads. This may suggest an issue with data races when classifying test data, as training data values and objects across all threads remains the same. However, the test utilizing 128 threads maintains the highest accuracy percentage of 90.47\%, followed by the sequential implementation with an accuracy percentage of 89.88\%. 

\begin{figure}[htbp]
\centerline{\includegraphics{Fig3.png}}
\caption{Classification accuracy results from 1--1024 threads.}
\label{fig}
\end{figure}

\begin{figure}[htbp]
\centerline{\includegraphics{Fig4.png}}
\caption{Classification precision results from 1--1024 threads.}
\label{fig}
\end{figure}

Figure 4 illustrates the classification precision of 10 iterations across all tests. To clarify what precision is specifically for this paper, precision is the consistency of classification results across multiple runs. Therefore I am recording the ability to reproduce the same results across 10 iterations of the same input data. As we can see, there is once again a dramatic drop in precision between 2 and 32 threads. This only confirms my suspicions of data races occuring, but being more prevalent between tests with threads 2-32. If there was no data races, the precision would be 1.00 or close to 1.00, as the data given to my cosine similarity function would be the same every time. However, besides the drastic differences in precision between threads 2-32, we can see the sequential test and the test with 128 threads have a precision of 1.00. These tests generate the same results every single time across 10 iterations.

It is also worth noting that the sequential implementation, on average across all 10 tests, took approximately 7.8 minutes to complete all tasks, while the implementation utilizing 128 threads took an average of 2.3 minutes to complete all tasks. These two configurations were chosen for direct comparison because they were the only ones that consistently achieved a precision of 1.00, indicating deterministic behavior and the absence of data races. Furthermore, both tests yielded the highest classification accuracies across all tests, with the sequential version achieving 89.88\% and the 128-thread version achieving a slightly higher accuracy of 90.47\%. This comparison demonstrates that the sequential implementation is about 3.4 times slower than my parallel implementation with 128 threads, which not only maintained precision but marginally improved classification accuracy.

\subsection{Conclusion}
I present a parallel TF-IDF implementation in C++. I discovered significant performance improvements in the parallel implementation of the sequential implementation. While I also discovered some data race concerns amongst a certain number of threads, I also discovered utilizing 128 threads was optimal for performance, accuracy, and precision. My implementation uses both atomic operations and locking mechanisms, such as mutex locks. It allows for the classification of test data, provided there was training data given. I compared the parallel implementation to my sequential implementation using custom test scripts and found the optimal number of threads across all tasks for performance, classification accuracy, and precision was 128 threads. I plan to further develop this project and incorporate more functions and usages to further expand C++’s usage in modern NLP.

\begin{thebibliography}{00}
\bibitem{b1} G. Juraev and O. Bozorov, ``Using TF-IDF in text classification,'' AIP Conference Proceedings, vol. 2789, pp. 050017–-050017, Jan. 2023, \url{https:://doi.org/10.1063/5.0145520}.
\bibitem{b2} K. Park, J. S. Hong, and W. Kim, ``A Methodology Combining Cosine Similarity with Classifier for Text Classification,'' Applied Artificial Intelligence, vol. 34, no. 5, pp. 396--411, Feb. 2020, \url{https://doi.org/10.1080/08839514.2020.1723868}.
\bibitem{b3} R. R. Curtin \textit{et al.}, ``mlpack 4: a fast, header-only C++ machine learning library,'' \textit{Journal of Open Source Software}, vol. 8, no. 85, p. 5011, 2023, \url{https://doi.org/10.21105/joss.05011}.
\bibitem{b4} F. Pedregosa, G. Varoquaux, A. Gramfort, V. Michel, B. Thirion, O. Grisel, M. Blondel, P. Prettenhofer, R. Weiss, V. Dubourg, J. Vanderplas, A. Passos, D. Cournapeau, M. Brucher, M. Perrot, and E. Duchesnay, ``Scikit-learn: Machine Learning in Python,'' \textit{Journal of Machine Learning Research}, vol. 12, pp. 2825--2830, 2011.
\bibitem{b5} B. Madden, ``Oleander Stemming Library,'' May 19, 2024. Accessed Feb. 11, 2025. [Online]. Available: \url{https://github.com/Blake-Madden/OleanderStemmingLibrary}.
\bibitem{b6} B. Bose, ``BBC News Classification,'' 2019. Accessed Jan. 2025. [Online]. Available: \url{https://kaggle.com/competitions/learn-ai-bbc}.
\bibitem{b7} C. Sammut and G. I. Webb, eds., \textit{Encyclopedia of Machine Learning}, Springer US, Boston, MA, 2010, pp. 986--987. \url{https://doi.org/10.1007/978-0-387-30164-8_832}{10.1007/978-0-387-30164-8\_832}.
\end{thebibliography}
% \vspace{12pt}
% \color{red}
% IEEE conference templates contain guidance text for composing and formatting conference papers. Please ensure that all template text is removed from your conference paper prior to submission to the conference. Failure to remove the template text from your paper may result in your paper not being published.

% \clearpage
\section*{Appendix}
% \appendix

\begin{algorithm}
    \caption{Merge Document TF-IDF into Category creating combined TF-IDF scores.}
    \begin{algorithmic}[1]
    \State Initialize $word\_count$ as empty map
    \State $i$ = $0$
    \ForAll{$(word, tfidf) \in doc\_tf\_idf$}
        \State $i$ = $i + 1$
        \If{$word \in tf\_idf\_all$}
            \State tf\_idf\_all[word] = tf\_idf\_all[word] + $tfidf$
            \State word\_count[word] = word\_count[word] + $1$
        \Else
            \State tf\_idf\_all[word] = $tfidf$
            \State word\_count[word] = $1$
        \EndIf
    \EndFor
    \ForAll{$(word, count) \in word\_count$}
        \State tf\_idf\_all[word] = tf\_idf\_all[word] / $i$
    \EndFor
    \end{algorithmic}
\end{algorithm}

\begin{algorithm}
    \caption{Loads \textit{x} Document objects to a thread}
    \begin{algorithmic}[1]
    \State n\_docs\_th = n\_docs / num\_ths
    \State n\_last = n\_docs \% n\_docs\_th
    \State n\_last += n\_docs\_th
    \If{$n\_docs\_th$ == $1$}
        \State num\_ths\_used = n\_docs
    \Else
        \State num\_ths\_used = num\_ths
    \EndIf
    \For{$i = 0$ to $n\_docs$ step $n\_docs\_th$}
        \State \textbf{create\_thread}(function()
        \If{$i = n\_docs - n\_docs\_th$ \textbf{and} $n\_last > 0$}
            \For{$x = 0$ to $n\_last$}
                \If{$x + i < n\_docs$}
                    \State function($documents[x + i]$)
                \EndIf
            \EndFor
        \Else
            \For{$x = 0$ to $n\_docs\_th$}
                \If{$x + i < n\_docs$}
                    \State function($documents[x + i]$)
                \EndIf
            \EndFor
        \EndIf
        \State ) \Comment{End of lambda function}
    \EndFor
    \end{algorithmic}
\end{algorithm}

% \bigskip
    

\end{document}


% 
% \subsection{Units}
% \begin{itemize}
% \item Use either SI (MKS) or CGS as primary units. (SI units are encouraged.) English units may be used as secondary units (in parentheses). An exception would be the use of English units as identifiers in trade, such as ``3.5-inch disk drive''.
% \item Avoid combining SI and CGS units, such as current in amperes and magnetic field in oersteds. This often leads to confusion because equations do not balance dimensionally. If you must use mixed units, clearly state the units for each quantity that you use in an equation.
% \item Do not mix complete spellings and abbreviations of units: ``Wb/m\textsuperscript{2}'' or ``webers per square meter'', not ``webers/m\textsuperscript{2}''. Spell out units when they appear in text: ``. . . a few henries'', not ``. . . a few H''.
% \item Use a zero before decimal points: ``0.25'', not ``.25''. Use ``cm\textsuperscript{3}'', not ``cc''.)
% \end{itemize}
% 
% \subsection{Equations}
% Number equations consecutively. To make your 
% equations more compact, you may use the solidus (~/~), the exp function, or 
% appropriate exponents. Italicize Roman symbols for quantities and variables, 
% but not Greek symbols. Use a long dash rather than a hyphen for a minus 
% sign. Punctuate equations with commas or periods when they are part of a 
% sentence, as in:
% \begin{equation}
% a+b=\gamma\label{eq}
% \end{equation}
% 
% Be sure that the 
% symbols in your equation have been defined before or immediately following 
% the equation. Use ``\eqref{eq}'', not ``Eq.~\eqref{eq}'' or ``equation \eqref{eq}'', except at 
% the beginning of a sentence: ``Equation \eqref{eq} is . . .''
% 
% \subsection{\LaTeX-Specific Advice}
% 
% Please use ``soft'' (e.g., \verb|\eqref{Eq}|) cross references instead
% of ``hard'' references (e.g., \verb|(1)|). That will make it possible
% to combine sections, add equations, or change the order of figures or
% citations without having to go through the file line by line.
% 
% Please don't use the \verb|{eqnarray}| equation environment. Use
% \verb|{align}| or \verb|{IEEEeqnarray}| instead. The \verb|{eqnarray}|
% environment leaves unsightly spaces around relation symbols.
% 
% Please note that the \verb|{subequations}| environment in {\LaTeX}
% will increment the main equation counter even when there are no
% equation numbers displayed. If you forget that, you might write an
% article in which the equation numbers skip from (17) to (20), causing
% the copy editors to wonder if you've discovered a new method of
% counting.
% 
% {\BibTeX} does not work by magic. It doesn't get the bibliographic
% data from thin air but from .bib files. If you use {\BibTeX} to produce a
% bibliography you must send the .bib files. 
% 
% {\LaTeX} can't read your mind. If you assign the same label to a
% subsubsection and a table, you might find that Table I has been cross
% referenced as Table IV-B3. 
% 
% {\LaTeX} does not have precognitive abilities. If you put a
% \verb|\label| command before the command that updates the counter it's
% supposed to be using, the label will pick up the last counter to be
% cross referenced instead. In particular, a \verb|\label| command
% should not go before the caption of a figure or a table.
% 
% Do not use \verb|\nonumber| inside the \verb|{array}| environment. It
% will not stop equation numbers inside \verb|{array}| (there won't be
% any anyway) and it might stop a wanted equation number in the
% surrounding equation.
% 
% \subsection{Some Common Mistakes}\label{SCM}
% \begin{itemize}
% \item The word ``data'' is plural, not singular.
% \item The subscript for the permeability of vacuum $\mu_{0}$, and other common scientific constants, is zero with subscript formatting, not a lowercase letter ``o''.
% \item In American English, commas, semicolons, periods, question and exclamation marks are located within quotation marks only when a complete thought or name is cited, such as a title or full quotation. When quotation marks are used, instead of a bold or italic typeface, to highlight a word or phrase, punctuation should appear outside of the quotation marks. A parenthetical phrase or statement at the end of a sentence is punctuated outside of the closing parenthesis (like this). (A parenthetical sentence is punctuated within the parentheses.)
% \item A graph within a graph is an ``inset'', not an ``insert''. The word alternatively is preferred to the word ``alternately'' (unless you really mean something that alternates).
% \item Do not use the word ``essentially'' to mean ``approximately'' or ``effectively''.
% \item In your paper title, if the words ``that uses'' can accurately replace the word ``using'', capitalize the ``u''; if not, keep using lower-cased.
% \item Be aware of the different meanings of the homophones ``affect'' and ``effect'', ``complement'' and ``compliment'', ``discreet'' and ``discrete'', ``principal'' and ``principle''.
% \item Do not confuse ``imply'' and ``infer''.
% \item The prefix ``non'' is not a word; it should be joined to the word it modifies, usually without a hyphen.
% \item There is no period after the ``et'' in the Latin abbreviation ``et al.''.
% \item The abbreviation ``i.e.'' means ``that is'', and the abbreviation ``e.g.'' means ``for example''.
% \end{itemize}
% An excellent style manual for science writers is \cite{b7}.
% 
% \subsection{Authors and Affiliations}
% \textbf{The class file is designed for, but not limited to, six authors.} A 
% minimum of one author is required for all conference articles. Author names 
% should be listed starting from left to right and then moving down to the 
% next line. This is the author sequence that will be used in future citations 
% and by indexing services. Names should not be listed in columns nor group by 
% affiliation. Please keep your affiliations as succinct as possible (for 
% example, do not differentiate among departments of the same organization).
% 
% \subsection{Identify the Headings}
% Headings, or heads, are organizational devices that guide the reader through 
% your paper. There are two types: component heads and text heads.
% 
% Component heads identify the different components of your paper and are not 
% topically subordinate to each other. Examples include Acknowledgments and 
% References and, for these, the correct style to use is ``Heading 5''. Use 
% ``figure caption'' for your Figure captions, and ``table head'' for your 
% table title. Run-in heads, such as ``Abstract'', will require you to apply a 
% style (in this case, italic) in addition to the style provided by the drop 
% down menu to differentiate the head from the text.
% 
% Text heads organize the topics on a relational, hierarchical basis. For 
% example, the paper title is the primary text head because all subsequent 
% material relates and elaborates on this one topic. If there are two or more 
% sub-topics, the next level head (uppercase Roman numerals) should be used 
% and, conversely, if there are not at least two sub-topics, then no subheads 
% should be introduced.
% 
% \subsection{Figures and Tables}
% \paragraph{Positioning Figures and Tables} Place figures and tables at the top and 
% bottom of columns. Avoid placing them in the middle of columns. Large 
% figures and tables may span across both columns. Figure captions should be 
% below the figures; table heads should appear above the tables. Insert 
% figures and tables after they are cited in the text. Use the abbreviation 
% ``Fig.~\ref{fig}'', even at the beginning of a sentence.
% 
% \begin{table}[htbp]
% \caption{Table Type Styles}
% \begin{center}
% \begin{tabular}{|c|c|c|c|}
% \hline
% \textbf{Table}&\multicolumn{3}{|c|}{\textbf{Table Column Head}} \\
% \cline{2-4} 
% \textbf{Head} & \textbf{\textit{Table column subhead}}& \textbf{\textit{Subhead}}& \textbf{\textit{Subhead}} \\
% \hline
% copy& More table copy$^{\mathrm{a}}$& &  \\
% \hline
% \multicolumn{4}{l}{$^{\mathrm{a}}$Sample of a Table footnote.}
% \end{tabular}
% \label{tab1}
% \end{center}
% \end{table}
% 
% \begin{figure}[htbp]
% \centerline{\includegraphics{fig1.png}}
% \caption{Example of a figure caption.}
% \label{fig}
% \end{figure}
% 
% Figure Labels: Use 8 point Times New Roman for Figure labels. Use words 
% rather than symbols or abbreviations when writing Figure axis labels to 
% avoid confusing the reader. As an example, write the quantity 
% ``Magnetization'', or ``Magnetization, M'', not just ``M''. If including 
% units in the label, present them within parentheses. Do not label axes only 
% with units. In the example, write ``Magnetization (A/m)'' or ``Magnetization 
% \{A[m(1)]\}'', not just ``A/m''. Do not label axes with a ratio of 
% quantities and units. For example, write ``Temperature (K)'', not 
% ``Temperature/K''.
% 
% \section*{Acknowledgment}
% 
% The preferred spelling of the word ``acknowledgment'' in America is without 
% an ``e'' after the ``g''. Avoid the stilted expression ``one of us (R. B. 
% G.) thanks $\ldots$''. Instead, try ``R. B. G. thanks$\ldots$''. Put sponsor 
% acknowledgments in the unnumbered footnote on the first page.
% 
% \section*{References}
% 
% Please number citations consecutively within brackets \cite{b1}. The 
% sentence punctuation follows the bracket \cite{b2}. Refer simply to the reference 
% number, as in \cite{b3}---do not use ``Ref. \cite{b3}'' or ``reference \cite{b3}'' except at 
% the beginning of a sentence: ``Reference \cite{b3} was the first $\ldots$''
% 
% Number footnotes separately in superscripts. Place the actual footnote at 
% the bottom of the column in which it was cited. Do not put footnotes in the 
% abstract or reference list. Use letters for table footnotes.
% 
% Unless there are six authors or more give all authors' names; do not use 
% ``et al.''. Papers that have not been published, even if they have been 
% submitted for publication, should be cited as ``unpublished'' \cite{b4}. Papers 
% that have been accepted for publication should be cited as ``in press'' \cite{b5}. 
% Capitalize only the first word in a paper title, except for proper nouns and 
% element symbols.
% 
% For papers published in translation journals, please give the English 
% citation first, followed by the original foreign-language citation \cite{b6}.
% 